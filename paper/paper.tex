\documentclass{article}
\usepackage[round,comma,]{natbib}
\usepackage{fullpage}
\usepackage{authblk}
\usepackage{graphicx}
\usepackage{minted}
\usepackage{tcolorbox}
\usepackage{etoolbox}
\BeforeBeginEnvironment{minted}{\begin{tcolorbox}}
\AfterEndEnvironment{minted}{\end{tcolorbox}}
\usepackage{url}
\usepackage{fancyvrb}

% local definitions
\newcommand{\msprime}[0]{\texttt{msprime}}
\newcommand{\stdpopsim}[0]{\texttt{stdpopsim}}
\newcommand{\demes}[0]{\texttt{demes}}

\newcommand{\aprcomment}[1]{{\textcolor{blue}{APR: #1}}}

\begin{document}

\title{\demes: }
\author[1]{Graham Gower}
\author[1]{Jerome Kelleher}
\author[1]{Aaron P. Ragsdale}
\author[1]{Kevin Thornton}
\affil[1]{Authors listed alphabetically}
\maketitle

\abstract{
}

\section*{Introduction}

\begin{itemize}
  \item The problem (see also \citet{ragsdale2020lessons})
  \begin{itemize}
    \item Simulation is central in population genetics studies, which requires defining
      and implementing sometimes complex demographic scenarios
    \item From a software perspective, every simulation software has its own interface,
      making it difficult to jump between them, and they are often not easily
      human-readable
    \item From a implementation perspective, complixity of proposed or inferred demographic
      models has increased rapidly: complex models with many populations often means
      many parameters and demographic events to correctly order and keep track of
    \item Reproducibility: published inferred demographic models are shared and reused
      by the community, but they are often described in unfriendly ways (maybe just a
      short description with a table for parameters), requiring someone to have to
      decipher and then attempt to reimplement the model themselves. An unambiguously
      coded and human-readable description would go a long way to alleviating these pains.
      \citep{adrion2020community}
    \item \citep{gutenkunst2009inferring,thornton2014c++,kelleher2016efficient,jouganous2017inferring}
  \end{itemize}
  \item Examples
  \begin{itemize}
    \item Simple example: OOA \citep{gutenkunst2009inferring,jouganous2017inferring}
    \item Complex example: Browning?
    \item Non-human: maybe with non-standard mating feature like inbreeding?
  \end{itemize}
  \item Summary of how our software addresses these issues
\end{itemize}

\section*{\demes: xxx}
Usage, features, etc...

\section*{Prospects and future developments}

\bibliographystyle{unsrtnat}
\bibliography{paper}

\end{document}
